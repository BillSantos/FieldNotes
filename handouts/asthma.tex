% Anesthesiology FieldNotes by William C. Santos is licensed under a
% Creative Commons Attribution-ShareAlike 3.0 Unported License.
% http://creativecommons.org/licenses/by-sa/3.0/deed.en_US
%
% Creative Commons Attribution Notice:
% Portions of this file courtesy of Mark L. Palmeri, M.D., Ph.D.
% https://github.com/mlp6/Intro-Medical-Instrumentation
%\documentclass[11pt]{exam}
\documentclass[11pt,letterpaper,draft]{exam}
\usepackage{epsf}
\usepackage{amsmath}
\usepackage{amssymb}
\usepackage{palatino}
\usepackage[dvips]{graphics}
% The following relax commands are needed to avoid conflict between
% the Exam class (exam.cls) and the fancyhdr (fancyhrd.sty) package:
\let\lhead\relax
\let\chead\relax
\let\rhead\relax
\let\lfoot\relax
\let\cfoot\relax
\let\rfoot\relax
% Now it's OK to use fancyhdr with the Exam class.
\usepackage{fancyhdr}
\usepackage{hyperref}
\parindent 0in
\parskip 1ex
\oddsidemargin  0in
\evensidemargin 0in
\textheight 8.5in
\textwidth 6.5in
\topmargin -0.25in


% Settings for the Exam Class Environment
\usepackage{ifpdf}
\printanswers
\unframedsolutions
\addpoints
%Change question parts to be enumerated w/o parenthesis:
\renewcommand{\partlabel}{\thepartno .}
% Defines the title of the solution environment:
\renewcommand{\solutiontitle}{\noindent\textbf{}\noindent}
\renewcommand{\SolutionEmphasis}{\itshape \small}

\pagestyle{fancy}
\fancyhead[LO]{\bf BME354L - Palmeri - Spring 2013}
\fancyhead[RO]{{\bf Second-Order Systems}}
\fancyfoot[C]{This work is licensed under a Creative Commons Attribution 3.0 Unported License.}


\begin{document}


% The next 7 lines were used when printing this section as a handout:
\setcounter{page}{1}
\setcounter{equation}{0}
\setcounter{figure}{0}
\begin{center}
\Large{\textbf{Asthma}}
\normalsize
\end{center}

\section*{Introduction}

\textit{Recall} that asthma affects over 300 million people worldwide, and that
this number is increasing. It accounts for over 250,000 deaths annually.

As with other respiratory diseases, Asthma places patients at risk for
increased perioperative morbidity and mortality, as well as increased hospital
length of stay.

It occurs about 20 percent more frequently in women, and is caused by both
genetic and external factors.

that allows the conversion of information between two different
domains, i.e.\ the \text{time} and \text{frequency} domains.

The Fourier transform is another such mathematical
transformation, operating on signals of the form $Ae^{j(\omega t +\theta)}$ and combinations thereof.

Although these types of signals represent just
a subset of what can be described with the Laplace transform, this
subset's importance makes the Fourier transform a ubiquitous analysis
method. 

While not every mathematical function has a Fourier transform, any
physically realizable signal \textit{does}.  That's good enough for us!

\subsubsection*{Pathogeneis and Diagnosis}

There are a number of reasons this is important, although they may not
be apparent until you begin applying the transform in practical
situations.

Some important mathematical operations are easier to perform in the
frequency domain.

\begin{itemize}
\item The significance of various components of a signal may be more obvious
in the frequency domain.
\item The Fourier transform converts a signal into time domain to its
equivalent in the frequency domain.
\end{itemize}

Why are we interested in the frequency domain?

\begin{enumerate}
\item If we are interested in the frequency content of a signal, the
frequency domain representation is an intuitive way of display this
information.
\item Some useful signal processing operations are much easier to perform
in the frequency domain. This applies both to intuition and to
processing efficiency, through the development of what are known as
\textit{fast Fourier transform} (FFT)
algorithms.
\end{enumerate}

The Fourier transform and its inverse, while perhaps sounding
intimidating, are simply integrals:

\begin{equation}
\begin{split}
G(\omega) =& \int^{\infty}_{-\infty}g(t)\exp(-j\omega t)~ dt\\
g(t) =& \frac{1}{2\pi}\int^{\infty}_{-\infty}G(\omega)\exp(j\omega t)~ d\omega\\
\end{split}
\end{equation}

These equations represent the decomposition and reconstruction of a
signal into components of the form  $Ae^{j(\omega t + \theta)}$, known
as the transform's \textit{basis functions}.

\subsubsection*{What the \%*\$\&\# does all this mean?}

The forward transform answers a pair of questions:

\textit{Is a sinusoidal wave of a particular frequency 
$\omega$ present in a signal?}

\textit{If so, with what amplitude $A$ and phase $\theta$?}

\begin{itemize}
\item A particular solution of the forward Fourier transform integral
is thus a
single complex number for a single frequency.  

\item This solution is the complex coefficient $Ae^{j \theta}$ of the complex exponential $e^{j\omega t}$.

\item To determine the total content of a signal i.e.\ all of its frequency
components, this integral must be repeatedly calculated, at every frequency!

\item Fortunately, if the signal can be described in symbolic form (as
a general equation), its
transform can often be likewise expressed in symbolic form.
\end{itemize}

\subsubsection*{How does the integral work?}

The \textbf{forward} Fourier transform utilizes an important
characteristic of signals of the form $e^{j\omega}$ known as \textit{orthogonality}:

\begin{equation*}
G(\omega)= \int^{\infty}_{-\infty}\exp(j\omega_1
t)\exp(-j\omega_2 t)~dt = 2\pi \text{ when }\omega_1=\omega_2, 0
\text{ otherwise}.
\end{equation*}

\begin{itemize}
\item In non-technical terms, the integral over all time of the
conjugate product of complex exponentials at two
different frequencies is always zero.

\item Thus, when computing the Fourier integral at a particular frequency,
the component of the signal at that frequency will be ``sifted'' out
and its amplitude and phase identified.

\item Analyzing a signal with the Fourier transform is akin to describing a
``recipe'' for the signal, listing the magnitude and phase of components of the signal as a
function of frequency.

\item In this light, the \textit{inverse} Fourier transform can be seen as taking
that recipe and using it to build up the original signal again. 

\item A particular solution of the inverse integral provides the value of
the signal in question \textit{at a single time}.

\end{itemize}
% \textit{This would be a good time to review the complex exponential
% and Euler's formula, which is required to solve the Fourier transform
% of the rectangle function.}

\subsubsection*{Calculating the integral}
We can't put it off any longer!  It's time to compute the Fourier
transform of something.  

Let's start at the very beginning, with $\cos(\omega t)$:

\begin{equation}
\begin{split}
\text{Recall}~~\cos(\omega_c t) &= \frac{1}{2}[\exp(j\omega_c t)+\exp(-j\omega_c t)]\\
G(\omega) =& \int^{\infty}_{-\infty}\cos(\omega_c t)\exp(-j\omega t)~dt\\
=& \int^{\infty}_{-\infty}\frac{1}{2}[\exp(j\omega_c t)+\exp(-j\omega_c
t)]\exp(-j\omega t)~dt\\
=& 2\pi\frac{1}{2}\int^{\infty}_{-\infty}\exp(j\omega_c t)\exp(-j\omega
t)~dt + 2\pi\frac{1}{2}\int^{\infty}_{-\infty}\exp(-j\omega_c t)\exp(-j\omega
t)~dt\\
=& 2\pi\frac{1}{2}[\delta(\omega+\omega_c)+\delta(\omega-\omega_c)] = 
 \pi[\delta(\omega+\omega_c)+\delta(\omega-\omega_c)]
\end{split}
\end{equation}

In this solution, we introduced an important function known as the
\textit{delta} function:\\
$\delta(x) = 1 \text { for } x=0, ~~0$ otherwise.

The orthogonality principle is at work here, in that the Fourier
integrals above are zero except at the frequencies $\exp(j\omega_c t)$
and $\exp(-j\omega_c t)$.

\textbf{But wait! The original signal seems to only have \textit{one} frequency
component in it, namely $\cos(\omega t)$, but the Fourier transform
has \textit{two}.  What's going on?}

\begin{itemize}
\item The original signal is a solely real signal, while
the basis functions of the Fourier transform are complex-valued.

\item To construct a solely-real signal, we have to take two complex
exponentials of equal frequency and magnitude but opposite phase and
add them together such at their imaginary parts \textit{cancel out}.
\end{itemize}

In light of this, we see Euler's cosine identity with a whole new perspective:
\begin{equation*}
\cos \omega t = \frac{e^{j \omega t} + e^{-j \omega t}}{2}
\end{equation*}

\begin{figure*}[ht]
\centering
\epsfxsize=5.0in
% \epsfbox{cos_ft.eps}
\caption{The Fourier transform of $\cos(\omega t)$ in magnitude and phase.}
\label{fig:cos_ft}
\end{figure*}

Some observations: 
\begin{itemize}
\item $e^{j \omega t}$ and $e^{-j \omega t}$ are both phasors, rotating
counter-clockwise and clockwise. 
\item The frequency domain is divided into
both signs of the exponent, referred to as the positive and negative
frequencies. 
\item There is one unique frequency that is neither positive nor
negative at $\omega
= 0$, corresponding to the constant mean value of the signal.  
\item This is typically
referred to as the ``DC'' component even though the signals involved are often not
current signals.
\end{itemize}

Let's move to the next logical choice of a fundamental function, with $\sin(\omega t)$:
\begin{equation}
\begin{split}
G(\omega) =& \int^{\infty}_{-\infty}\sin(\omega_c t)\exp(-j\omega t)~dt\\
=& \int^{\infty}_{-\infty}\frac{1}{2j}[\exp(j\omega_c t)-\exp(-j\omega_c
t)]\exp(-j\omega t)~dt\\
=& 2\pi\frac{1}{2j}\int^{\infty}_{-\infty}\exp(j\omega_c t)\exp(-j\omega
t)~dt - 2\pi\frac{1}{2j}\int^{\infty}_{-\infty}\exp(-j\omega_c t)\exp(-j\omega
t)~dt\\
=& 2\pi\frac{1}{2j}[\delta(\omega-\omega_c)-\delta(\omega+\omega_c)] =
\frac{\pi}{j}[\delta(\omega-\omega_c)-\delta(\omega+\omega_c)]
\end{split}
\end{equation}

Once again, we can relate this result to one of Euler's identities,
this time for sine:

\begin{equation*}
\sin \omega t = \frac{e^{j \omega t} - e^{-j \omega t}}{2j}
\end{equation*}

This result is the same as for cosine in magnitude, but different in
phase.  

This makes intuitive sense, because the only difference between a sine
wave and a cosine wave of the same frequency and amplitude is their
\textit{phase}.

%\begin{figure}[ht]
%\centering
%\epsfxsize=5.0in
%\epsfbox{sin_ft.eps}
%\caption{The Fourier transform of $\sin(\omega t)$ in magnitude and phase.}
%\label{fig:sin_ft}
%\end{figure}

Let's consider the transform of the \textit{rectangle} function, defined
as $x/2$ for $|x|<1$:
\begin{equation}
\begin{split}
G(\omega) =& \int^{\infty}_{-\infty}\text{rect}(t)\exp(-j\omega t)~dt\\
=& \frac{1}{2}\int^{1}_{-1}\exp(-j\omega t)~dt = 
\frac{\exp(-j\omega t)}{-j\omega}\mid^{1}_{-1}\\
=& \frac{\exp(-j\omega)-\exp(j\omega)}{-j2\omega}\\
=& \frac{\sin\omega}{\omega}
\end{split}
\end{equation}

This result is known as the $sinc$ function, where $sinc(\omega) =  \frac{\sin\omega}{\omega}$.

\textit{As in the case of the Laplace transform, the Fourier
transforms of many fundamental functions are listed in reference tables.}

\subsubsection*{Symmetry: evenness and oddness}

The cosine and sine functions exhibit a pair of properties that are
important to understand, as they are frequently mentioned in
discussions of the Fourier transform and its applications.

The function $\cos(\omega t)$ is what is known as an \textit{even}
function, which has form $f(t) = -f(t)$. An even function is thus
\textit{symmetric} about the origin.

The function $\sin(\omega t)$ is what is known as an \textit{odd}
function, which has form $f(t) = -f(-t)$. An even function is thus
\textit{anti-symmetric} about the origin.

\subsection*{Fourier series}

A function and it Fourier transform solution constitute a Fourier
transform \textit{pair.}

Before considering more Fourier transform pairs, lets consider a
special case of the transform, which is that for periodic signals of
finite power, and which produces a \textit{Fourier series}.

A periodic signal of period $T$ has form $f(t+nT) = f(t)$, $n$ an integer.  

\begin{itemize}
\item It turns out that a periodic signal of finite power can be decomposed
into what is known as a \textit{harmonic series}.
\item This is the weighted
combination of a orthogonal signals with the same period, beginning
with the \textit{fundamental} frequency and integer multiples of this frequency
known as the \textit{harmonics}.

\item Thus, a repetitive or periodic signal will have a discrete frequency
spectrum.

\item (Note that sometimes the fundamental is called the \textit{first harmonic}.
This is just a convention, like calling the ground floor of a building
the first floor.)

\item For such signals, the Fourier integral can be calculated over a single
period of the signal rather than over all time.

\end{itemize}

Signals $g(t)$ and $f(t)$, each with period $T$, are orthogonal if
\begin{equation*}
\int^{T/2}_{-T/2}f(t)g(t)dt = 0
\end{equation*}

The Fourier series decomposition of the signal $f(t)$ has the form:
\begin{equation*}
f(t) = a_{0}/2 + \sum^{\infty}_{n=1}[a_{n}\cos(n\omega_{0}t)+b_{n}\sin(n\omega_{0}t)]
\end{equation*}

This weighted combination of sines and cosines is governed by the coefficients
$a_{n}$ and $b_{n}$, calculated with integrals similar to the Fourier integral:
\begin{equation*}
\begin{split}
a_{n}&=\frac{2}{T_{0}}\int^{T/2}_{-T/2}f(t)\cos(n\omega_{0}t)dt\\
b_{n}&=\frac{2}{T_{0}}\int^{T/2}_{-T/2}f(t)\sin(n\omega_{0}t)dt
\end{split}
\end{equation*}

\begin{itemize}
\item $a_{0}/2 =$  the mean, or ``DC,'' component of signal. $\omega_{0} = 0$ at DC.

\item Even signals comprise cosine terms only ($b_{n}=0$ for all $n$).

\item Odd signals comprise sine terms only ($a_{n}=0$ for all $n$).
\end{itemize}

If you can determine the coefficients of a Fourier series
representation of the signal, the magnitude-squared value of these
coefficients indicates the energy associated with each discrete
frequency component in the series.

Note that the spectrum of periodic signals is always discrete!

Let's calculate the Fourier series solution for a square wave:
\begin{equation*}
\begin{split}
f(t+nT) & =  -1 \text{ for } T/2 < t \le 0, 1 \text{ for }0<t\le T/2\\
a_{n}&=-\frac{2}{T_{0}}\int^{0}_{-T/2}\cos(n\omega_{0}t)dt+\frac{2}{T_{0}}\int^{T/2}_{0}\cos(n\omega_{0}t)dt\\
b_{n}&=-\frac{2}{T_{0}}\int^{0}_{-T/2}\sin(n\omega_{0}t)dt+\frac{2}{T_{0}}\int^{T/2}_{0}\sin(n\omega_{0}t)dt\\
a_{n}&= 0 \text{ by inspection for cosine, an even function,}\\
b_{n}&=\frac{4}{T_{0}}\int^{T/2}_{0}\sin(n\omega_{0}t)dt\text{ by inspection for sine, an odd function,}\\
b_{n}&=-\frac{4}{nT_{0}\omega_{0}}\cos(n\omega_{0}t)\mid^{T/2}_{0}\\
b_{n}&=-\frac{4}{nT_{0}\omega_{0}}\bigg{(}\cos(n\omega_{0}T/2)-1\bigg{)}\\
b_{n}&= 0 \text{ for even integer n,}\\
b_{n}&= \frac{8}{nT_{0}\omega_{0}} \text{ for odd integer n,}\\
f(t)&=\sum^{\infty}_{n=1,3,5...}\frac{8}{nT_{0}\omega_{0}}=\sum^{\infty}_{n=1,3,5...}\frac{4}{n\pi}\text{
since }\omega_{0}=\frac{2\pi}{T_{0}}\\
\end{split}
\end{equation*}
One observes that a square wave is the sum of the odd integer sine harmonics.

This results is illustrated in Figure~\ref{fig:square_fs}.

A certain degree of overshoot is evident in the reconstruction.  This
is present even for a sum of a large number of harmonics, and is known as \textit{Gibb's phenomenon.}

\begin{figure}[ht]
\centering
\epsfxsize=6.0in
% \epsfbox{square_fs.eps}
\caption{The Fourier series deconstruction of a square wave is illustrated, with
sine component amplitudes (the function's \textit{discrete magnitude
spectrum}) shown in the top figure and series sums over
time shown for $n = 1$, $n = [1:5]$, and $n=[1:35]$.}  
\label{fig:square_fs}
\end{figure}

Let's consider the Fourier series solution of a sequence of delta
functions known as the \textit{comb} function:
\begin{equation*}
\begin{split}
f(t) & = comb(t) = \sum^{\infty}_{n=-\infty}\delta(t-n\tau)\\
a_{n}&=\frac{2}{T_{0}}\cos(0)=\frac{2}{T_{0}}\\
f(t)&=\frac{1}{T_{0}}+\sum^{\infty}_{n=-\infty}\frac{2}{T_{0}}\cos
n\omega_{0}t\\
\omega_{0}&=\frac{2\pi}{T_{0}}\\
\therefore F(\omega)& = \frac{2\pi}{T_{0}}comb(\omega)
\end{split}
\end{equation*}

Thus, the Fourier transform of a $comb$ is a $comb$!
\begin{itemize}
\item Multiplication of a signal by the $comb$ function is used to represent
the sampling operation of analog to digital conversion.
\item We will find that the digitally sampled signal will have a periodic spectrum.
\item A general observation is that sampling in one of domain produces
periodicity in the other domain.
\end{itemize}

\subsection*{Fourier transform pairs}

\begin{equation*}
\begin{split}
f(t) = A\cos(\omega_c t) &\Longleftrightarrow F(\omega) = A\pi[\delta(\omega-\omega_c)+\delta(\omega+\omega_c)]\\
f(t) = A\sin(\omega_c t)&\Longleftrightarrow F(\omega) = \frac{A\pi}{j}[\delta(\omega-\omega_c)-\delta(\omega+\omega_c)]\\
f(t) = rect(t) &\Longleftrightarrow F(\omega) =
\frac{\sin\omega}{\omega} = sinc(\omega)\\
f(t) = A &\Longleftrightarrow F(\omega) = 2\pi A\delta(\omega)\\
f(t) = \delta(t-t_0) &\Longleftrightarrow F(\omega) = \exp(-j\omega
t_0),~ = 1 \text{ for } t_0 = 0.
\end{split}
\end{equation*}

%----------------------------------
\subsection*{Properties of the Fourier transform}

Extensive tables of Fourier transform \textit{pairs} exist showing
fundamental functions and their transforms.

The Fourier transform exhibits a series of properties that can greatly
simplify analysis.  These properties also allow \textit{extension} of the
fundamental transform pairs by generalization.

Given $f(t)$ and its Fourier transform $F(\omega)$:

\begin{equation*}
\begin{split}
\text{Scaling:}~~f(at)~~&\Longleftrightarrow~~\frac{1}{|a|}F\bigg{(}\frac{\omega}{a}\bigg{)}\\
\text{Superposition:}~~af(t)+bg(t)~~&\Longleftrightarrow~~aF(\omega)+bG(\omega)\\
\text{Translation~/~shift:}~~f(t-t_0)~~&\Longleftrightarrow~~\exp\big{(}-j\omega
t_0\big{)}F(\omega)\\
\text{Convolution:}~~f(t)\ast
g(t)~~&\Longleftrightarrow~~F(\omega)~G(\omega)\\
\text{Autocorrelation:}~~f(t)\ast
f^\ast(-t)~~&\Longleftrightarrow~~|F(\omega)^2|\\
\text{Cross-correlation:}~~f(t)\ast
g^\ast(-t)~~&\Longleftrightarrow~~F(\omega)~G^{\ast}(\omega)\\
\\
\text{Power (Parseval's Theorem):}~~\int^{\infty}_{-\infty}f(t)g^{\ast}(t)~dx~~&=~~
\int^{\infty}_{-\infty}F(\omega)~G^{\ast}(\omega)~d\omega
\end{split}
%\label{}
\end{equation*}

Note that the scaling property has an amplitude term $1/|a|$ out
front.  \textit{For reasons beyond the scope of this discussion, this
factor is only included for functions having finite energy, such as a
square pulse.}

This amplitude scaling  does not apply to periodic signals.  Thus the
Fourier transform of $\cos(2\omega_0 t) = \pi[\delta(\omega +
2\omega_0) + \delta(\omega - 2\omega_0)]$, not $\pi/2[\delta(\omega +
2\omega_0) + \delta(\omega - 2\omega_0)]$.


\end{document}
